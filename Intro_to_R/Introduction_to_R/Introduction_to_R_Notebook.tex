\documentclass[]{article}
\usepackage{lmodern}
\usepackage{amssymb,amsmath}
\usepackage{ifxetex,ifluatex}
\usepackage{fixltx2e} % provides \textsubscript
\ifnum 0\ifxetex 1\fi\ifluatex 1\fi=0 % if pdftex
  \usepackage[T1]{fontenc}
  \usepackage[utf8]{inputenc}
\else % if luatex or xelatex
  \ifxetex
    \usepackage{mathspec}
  \else
    \usepackage{fontspec}
  \fi
  \defaultfontfeatures{Ligatures=TeX,Scale=MatchLowercase}
\fi
% use upquote if available, for straight quotes in verbatim environments
\IfFileExists{upquote.sty}{\usepackage{upquote}}{}
% use microtype if available
\IfFileExists{microtype.sty}{%
\usepackage{microtype}
\UseMicrotypeSet[protrusion]{basicmath} % disable protrusion for tt fonts
}{}
\usepackage[margin=1in]{geometry}
\usepackage{hyperref}
\hypersetup{unicode=true,
            pdftitle={INTRODUCTION TO R},
            pdfauthor={Richard Johansen},
            pdfborder={0 0 0},
            breaklinks=true}
\urlstyle{same}  % don't use monospace font for urls
\usepackage{color}
\usepackage{fancyvrb}
\newcommand{\VerbBar}{|}
\newcommand{\VERB}{\Verb[commandchars=\\\{\}]}
\DefineVerbatimEnvironment{Highlighting}{Verbatim}{commandchars=\\\{\}}
% Add ',fontsize=\small' for more characters per line
\usepackage{framed}
\definecolor{shadecolor}{RGB}{248,248,248}
\newenvironment{Shaded}{\begin{snugshade}}{\end{snugshade}}
\newcommand{\KeywordTok}[1]{\textcolor[rgb]{0.13,0.29,0.53}{\textbf{#1}}}
\newcommand{\DataTypeTok}[1]{\textcolor[rgb]{0.13,0.29,0.53}{#1}}
\newcommand{\DecValTok}[1]{\textcolor[rgb]{0.00,0.00,0.81}{#1}}
\newcommand{\BaseNTok}[1]{\textcolor[rgb]{0.00,0.00,0.81}{#1}}
\newcommand{\FloatTok}[1]{\textcolor[rgb]{0.00,0.00,0.81}{#1}}
\newcommand{\ConstantTok}[1]{\textcolor[rgb]{0.00,0.00,0.00}{#1}}
\newcommand{\CharTok}[1]{\textcolor[rgb]{0.31,0.60,0.02}{#1}}
\newcommand{\SpecialCharTok}[1]{\textcolor[rgb]{0.00,0.00,0.00}{#1}}
\newcommand{\StringTok}[1]{\textcolor[rgb]{0.31,0.60,0.02}{#1}}
\newcommand{\VerbatimStringTok}[1]{\textcolor[rgb]{0.31,0.60,0.02}{#1}}
\newcommand{\SpecialStringTok}[1]{\textcolor[rgb]{0.31,0.60,0.02}{#1}}
\newcommand{\ImportTok}[1]{#1}
\newcommand{\CommentTok}[1]{\textcolor[rgb]{0.56,0.35,0.01}{\textit{#1}}}
\newcommand{\DocumentationTok}[1]{\textcolor[rgb]{0.56,0.35,0.01}{\textbf{\textit{#1}}}}
\newcommand{\AnnotationTok}[1]{\textcolor[rgb]{0.56,0.35,0.01}{\textbf{\textit{#1}}}}
\newcommand{\CommentVarTok}[1]{\textcolor[rgb]{0.56,0.35,0.01}{\textbf{\textit{#1}}}}
\newcommand{\OtherTok}[1]{\textcolor[rgb]{0.56,0.35,0.01}{#1}}
\newcommand{\FunctionTok}[1]{\textcolor[rgb]{0.00,0.00,0.00}{#1}}
\newcommand{\VariableTok}[1]{\textcolor[rgb]{0.00,0.00,0.00}{#1}}
\newcommand{\ControlFlowTok}[1]{\textcolor[rgb]{0.13,0.29,0.53}{\textbf{#1}}}
\newcommand{\OperatorTok}[1]{\textcolor[rgb]{0.81,0.36,0.00}{\textbf{#1}}}
\newcommand{\BuiltInTok}[1]{#1}
\newcommand{\ExtensionTok}[1]{#1}
\newcommand{\PreprocessorTok}[1]{\textcolor[rgb]{0.56,0.35,0.01}{\textit{#1}}}
\newcommand{\AttributeTok}[1]{\textcolor[rgb]{0.77,0.63,0.00}{#1}}
\newcommand{\RegionMarkerTok}[1]{#1}
\newcommand{\InformationTok}[1]{\textcolor[rgb]{0.56,0.35,0.01}{\textbf{\textit{#1}}}}
\newcommand{\WarningTok}[1]{\textcolor[rgb]{0.56,0.35,0.01}{\textbf{\textit{#1}}}}
\newcommand{\AlertTok}[1]{\textcolor[rgb]{0.94,0.16,0.16}{#1}}
\newcommand{\ErrorTok}[1]{\textcolor[rgb]{0.64,0.00,0.00}{\textbf{#1}}}
\newcommand{\NormalTok}[1]{#1}
\usepackage{graphicx,grffile}
\makeatletter
\def\maxwidth{\ifdim\Gin@nat@width>\linewidth\linewidth\else\Gin@nat@width\fi}
\def\maxheight{\ifdim\Gin@nat@height>\textheight\textheight\else\Gin@nat@height\fi}
\makeatother
% Scale images if necessary, so that they will not overflow the page
% margins by default, and it is still possible to overwrite the defaults
% using explicit options in \includegraphics[width, height, ...]{}
\setkeys{Gin}{width=\maxwidth,height=\maxheight,keepaspectratio}
\IfFileExists{parskip.sty}{%
\usepackage{parskip}
}{% else
\setlength{\parindent}{0pt}
\setlength{\parskip}{6pt plus 2pt minus 1pt}
}
\setlength{\emergencystretch}{3em}  % prevent overfull lines
\providecommand{\tightlist}{%
  \setlength{\itemsep}{0pt}\setlength{\parskip}{0pt}}
\setcounter{secnumdepth}{0}
% Redefines (sub)paragraphs to behave more like sections
\ifx\paragraph\undefined\else
\let\oldparagraph\paragraph
\renewcommand{\paragraph}[1]{\oldparagraph{#1}\mbox{}}
\fi
\ifx\subparagraph\undefined\else
\let\oldsubparagraph\subparagraph
\renewcommand{\subparagraph}[1]{\oldsubparagraph{#1}\mbox{}}
\fi

%%% Use protect on footnotes to avoid problems with footnotes in titles
\let\rmarkdownfootnote\footnote%
\def\footnote{\protect\rmarkdownfootnote}

%%% Change title format to be more compact
\usepackage{titling}

% Create subtitle command for use in maketitle
\providecommand{\subtitle}[1]{
  \posttitle{
    \begin{center}\large#1\end{center}
    }
}

\setlength{\droptitle}{-2em}

  \title{INTRODUCTION TO R}
    \pretitle{\vspace{\droptitle}\centering\huge}
  \posttitle{\par}
    \author{Richard Johansen}
    \preauthor{\centering\large\emph}
  \postauthor{\par}
      \predate{\centering\large\emph}
  \postdate{\par}
    \date{10/31/2019}


\begin{document}
\maketitle

\subsection{Workshop Summary and Contact
Information}\label{workshop-summary-and-contact-information}

\textbf{Summary:} R is a free and powerful programming language that is
commonly used by researchers in both qualitative and quantitative
disciplines. R provides a near comprehensive, and still expanding set of
research and data analysis tools. This workshop provides a gradual
introduction to the basics of programming with R using R Studio. As a
participant in this workshop you will accomplish the following: explore
the R Studio interface, perform basic data manipulation using the
Tidyverse, learn how to install and run packages, and learn the basics
of programming with R. The focus of this workshop will be hands-on
exercises to provide a deeper and more effective understanding of R. No
programming experience is required, and beginners are encouraged to
attend.

\textbf{Contact:}\\
Email: \href{mailto:AskData@uc.edu}{\nolinkurl{AskData@uc.edu}}\\
Location: 240 Braunstein Hall (GMP Library)\\
Research \& Data Services Website:
\url{https://libraries.uc.edu/research-teaching-support/research-data-services.html}
GitHub: \url{https://github.com/RAJohansen/UCL_Workshops} Twitter:
\url{https://twitter.com/johansen_phd}

\subsubsection{Section I: Coding Basics}\label{section-i-coding-basics}

\paragraph{1. R as a Calculator/Working with Mathematical
Operators}\label{r-as-a-calculatorworking-with-mathematical-operators}

\subparagraph{Addition (+)}\label{addition}

\begin{Shaded}
\begin{Highlighting}[]
\DecValTok{1}\OperatorTok{+}\DecValTok{2} 
\end{Highlighting}
\end{Shaded}

\begin{verbatim}
## [1] 3
\end{verbatim}

\subparagraph{Subtraction (-)}\label{subtraction--}

\begin{Shaded}
\begin{Highlighting}[]
\DecValTok{15}\OperatorTok{-}\DecValTok{8}
\end{Highlighting}
\end{Shaded}

\begin{verbatim}
## [1] 7
\end{verbatim}

\subparagraph{Multiplication (*)}\label{multiplication}

\begin{Shaded}
\begin{Highlighting}[]
\DecValTok{12}\OperatorTok{*}\DecValTok{20}
\end{Highlighting}
\end{Shaded}

\begin{verbatim}
## [1] 240
\end{verbatim}

\subparagraph{Division (/)}\label{division}

\begin{Shaded}
\begin{Highlighting}[]
\DecValTok{20}\OperatorTok{/}\DecValTok{5}
\end{Highlighting}
\end{Shaded}

\begin{verbatim}
## [1] 4
\end{verbatim}

\subparagraph{Equation}\label{equation}

\begin{Shaded}
\begin{Highlighting}[]
\KeywordTok{sin}\NormalTok{(pi}\OperatorTok{*}\DecValTok{15}\NormalTok{)}\OperatorTok{/}\DecValTok{100}
\end{Highlighting}
\end{Shaded}

\begin{verbatim}
## [1] 5.389623e-17
\end{verbatim}

\paragraph{2. Objects and Assignment
Statements}\label{objects-and-assignment-statements}

R stores values and objects so they can be reused throughout an equation
or script

\subparagraph{Object Assignment}\label{object-assignment}

\begin{Shaded}
\begin{Highlighting}[]
\NormalTok{x <-}\StringTok{ }\DecValTok{1}\OperatorTok{+}\DecValTok{2}
\NormalTok{x}
\end{Highlighting}
\end{Shaded}

\begin{verbatim}
## [1] 3
\end{verbatim}

\subparagraph{Working with multiple
objects}\label{working-with-multiple-objects}

Hint alt - is a shortcut for the \textless{} -

\begin{Shaded}
\begin{Highlighting}[]
\NormalTok{y <-}\StringTok{ }\NormalTok{x }\OperatorTok{+}\DecValTok{1}
\NormalTok{y}
\end{Highlighting}
\end{Shaded}

\begin{verbatim}
## [1] 4
\end{verbatim}

\paragraph{3. Naming Convention}\label{naming-convention}

\subparagraph{Quick tips for naming
convention:}\label{quick-tips-for-naming-convention}

Short \& sweet\\
Case sensative: x != X\\
Must Start with a letter\\
Only numbers, letters, \_, and .\\
\#\#\#\#\# Cases: snake case: ``i\_use\_snake\_case''\\
Camel Case: ``iUseCamelCase''\\
periods: ``i.use.periods''\\
combinations: ``iCant\_decide.which\_toUse''\\
\#\#\#\# 4. Using Comments The \# sign is used to create a comment,
which are essential for reproducible code/ Remember, your most likely
collaborator is future you!!

\subparagraph{Simple example}\label{simple-example}

\begin{Shaded}
\begin{Highlighting}[]
\DecValTok{1}\OperatorTok{+}\DecValTok{1} \CommentTok{# This is the formula for 1+1}
\end{Highlighting}
\end{Shaded}

\begin{verbatim}
## [1] 2
\end{verbatim}

\subparagraph{More complicated}\label{more-complicated}

\begin{Shaded}
\begin{Highlighting}[]
\NormalTok{p <-}\StringTok{ }\FloatTok{3.14}
\NormalTok{RoC <-}\StringTok{ }\DecValTok{4}
\NormalTok{x <-}\StringTok{ }\NormalTok{p}\OperatorTok{*}\NormalTok{RoC}\OperatorTok{^}\DecValTok{2}
\end{Highlighting}
\end{Shaded}

\subparagraph{What is this calculating?}\label{what-is-this-calculating}

Can we rewrite this so others have an easier time understanding it?\\

\begin{Shaded}
\begin{Highlighting}[]
\NormalTok{pi <-}\StringTok{ }\FloatTok{3.14} \CommentTok{#rounded value of pi}
\NormalTok{r <-}\StringTok{ }\DecValTok{4} \CommentTok{#radius of a circle}
\NormalTok{area <-}\StringTok{ }\NormalTok{pi}\OperatorTok{*}\NormalTok{r}\OperatorTok{^}\DecValTok{2} \CommentTok{#formula for calculating the area of a circle}
\end{Highlighting}
\end{Shaded}

This still is somewhat intuitive but what if you're making your own
formula?\\
HINT: Do not use ``magic'' numbers\\
Every number standalone number should have a description Example p
above.\\
\#\#\#\# 5. Creating a series of numbers Lets say you want to create a
series of numbers from 1-10 and call the object ``series''\\
You can use the c function to combine elements together\\

\begin{Shaded}
\begin{Highlighting}[]
\NormalTok{series <-}\StringTok{ }\KeywordTok{c}\NormalTok{(}\DecValTok{1}\NormalTok{,}\DecValTok{2}\NormalTok{,}\DecValTok{3}\NormalTok{,}\DecValTok{4}\NormalTok{,}\DecValTok{5}\NormalTok{,}\DecValTok{6}\NormalTok{,}\DecValTok{7}\NormalTok{,}\DecValTok{8}\NormalTok{,}\DecValTok{9}\NormalTok{,}\DecValTok{10}\NormalTok{)}
\end{Highlighting}
\end{Shaded}

However what if you wanted to create a series to 100?\\
Using c() would take too long, so we use a range of values by using the
``:''\\

\begin{Shaded}
\begin{Highlighting}[]
\NormalTok{series <-}\StringTok{ }\DecValTok{1}\OperatorTok{:}\DecValTok{100}
\end{Highlighting}
\end{Shaded}

Easy right? This is because the series is in order but what if we want a
series of 1:100 by 2?\\
This is were we want to introduce functions

\paragraph{6. Understanding how to interpret
functions}\label{understanding-how-to-interpret-functions}

General recipe for functions:

\begin{Shaded}
\begin{Highlighting}[]
\KeywordTok{function_name}\NormalTok{(argument }\CommentTok{#1 = value #1,}
\NormalTok{              argument }\CommentTok{#2 = value #2)}
\end{Highlighting}
\end{Shaded}

Going back to our series task, we want to create a series of numbers
from 1 to 100 by 2. Luckily there are many functions already available
to use in base R (many many more available from packages, which we will
discuss later).\\[2\baselineskip]Given that we are just learning R, I
will tell you that the function is called ``seq()''\\
The first thing I do when using a new functions is to look at the
documentation. You can use the ? to find R documentation.\\
\textbf{HINT: Scroll to the bottom of the help page for workable
examples.}\\

\begin{Shaded}
\begin{Highlighting}[]
\NormalTok{?}\KeywordTok{seq}\NormalTok{()}
\end{Highlighting}
\end{Shaded}

\paragraph{7. Getting Help in R}\label{getting-help-in-r}

\textbf{HINT: if you can't remember exactly what function you are
looking for, Use Tab.}

\begin{Shaded}
\begin{Highlighting}[]
\NormalTok{me}\OperatorTok{<}\NormalTok{tab}\OperatorTok{>}
\end{Highlighting}
\end{Shaded}

Additionally, if you are not sure what the function is called try a
fuzzy search.\\

\begin{Shaded}
\begin{Highlighting}[]
\KeywordTok{apropos}\NormalTok{(}\StringTok{"mea"}\NormalTok{) }
\end{Highlighting}
\end{Shaded}

\subsubsection{Section I Tasks}\label{section-i-tasks}

Work with your neighbor or by yourself and explore the documentation to
be able to complete the following:

\paragraph{Task 1A:}\label{task-1a}

Create a sequence from 1 to 10 using the seq() function

\begin{Shaded}
\begin{Highlighting}[]
\KeywordTok{seq}\NormalTok{(}\DecValTok{1}\NormalTok{,}\DecValTok{10}\NormalTok{)}
\end{Highlighting}
\end{Shaded}

\begin{verbatim}
##  [1]  1  2  3  4  5  6  7  8  9 10
\end{verbatim}

\paragraph{Task 1B:}\label{task-1b}

Create a sequence of numbers from 50 to 150 by 3

\begin{Shaded}
\begin{Highlighting}[]
\KeywordTok{seq}\NormalTok{(}\DecValTok{50}\NormalTok{, }\DecValTok{150}\NormalTok{, }\DataTypeTok{by =}\DecValTok{3}\NormalTok{) }\CommentTok{#define step by value}
\end{Highlighting}
\end{Shaded}

\begin{verbatim}
##  [1]  50  53  56  59  62  65  68  71  74  77  80  83  86  89  92  95  98
## [18] 101 104 107 110 113 116 119 122 125 128 131 134 137 140 143 146 149
\end{verbatim}

\paragraph{Task 1C:}\label{task-1c}

Create a sequence of numbers from 25 to 0 decreasing by 5 and save it an
object called ``my\_seq''

\begin{Shaded}
\begin{Highlighting}[]
\NormalTok{my_seq <-}\StringTok{ }\KeywordTok{seq}\NormalTok{(}\DecValTok{25}\NormalTok{, }\DecValTok{0}\NormalTok{, }\DataTypeTok{by =}\OperatorTok{-}\DecValTok{5}\NormalTok{) }\CommentTok{#define step by value}
\end{Highlighting}
\end{Shaded}

\subsubsection{Section II: Creating, Importing and Exporting
data}\label{section-ii-creating-importing-and-exporting-data}

\paragraph{Data Types:}\label{data-types}

Vectors: vectors (a row of numbers, also called 1-dimensional arrays)\\
Matrices: Matrices are nothing more than 2-dimensional vectors.\\
Data Frames: A data frame is a matrix with names above the columns.\\
Lists: The main advantage of lists is that the columns or collection of
vectors, which don't have to be of the same length unlike matrices and
data frames.\\
\textbf{HINT: However most of the time you will be working with data
frames.}

\paragraph{1. Creating A data frame}\label{creating-a-data-frame}

We can create a data frame by combining multiple vectors together

\begin{Shaded}
\begin{Highlighting}[]
\NormalTok{employee <-}\StringTok{ }\KeywordTok{c}\NormalTok{(}\StringTok{'John'}\NormalTok{,}\StringTok{'Peter'}\NormalTok{,}\StringTok{'Abby'}\NormalTok{,}\StringTok{'Hope'}\NormalTok{,}\StringTok{'Robert'}\NormalTok{, }\StringTok{'Emily'}\NormalTok{)}
\NormalTok{salary <-}\StringTok{ }\KeywordTok{c}\NormalTok{(}\DecValTok{15000}\NormalTok{, }\DecValTok{23400}\NormalTok{, }\DecValTok{26800}\NormalTok{, }\DecValTok{22000}\NormalTok{,}\DecValTok{35000}\NormalTok{,}\DecValTok{45000}\NormalTok{)}
\NormalTok{experience <-}\StringTok{ }\KeywordTok{c}\NormalTok{(}\DecValTok{1}\NormalTok{,}\DecValTok{2}\NormalTok{,}\DecValTok{3}\NormalTok{,}\DecValTok{2}\NormalTok{,}\DecValTok{4}\NormalTok{,}\DecValTok{7}\NormalTok{)}
\end{Highlighting}
\end{Shaded}

Using the data.frame() function we can combine multiple vectors into a
data frame

\begin{Shaded}
\begin{Highlighting}[]
\NormalTok{company <-}\StringTok{ }\KeywordTok{data.frame}\NormalTok{(employee, salary, experience)}
\end{Highlighting}
\end{Shaded}

You can view the data by clicking in the upper right hand corrner on
Company, or use code through the View() function

\begin{Shaded}
\begin{Highlighting}[]
\KeywordTok{View}\NormalTok{(company)}
\end{Highlighting}
\end{Shaded}

Another way to explore your new data frame quickly is by using the
head() function.

\begin{Shaded}
\begin{Highlighting}[]
\KeywordTok{head}\NormalTok{(company)}
\end{Highlighting}
\end{Shaded}

\begin{verbatim}
##   employee salary experience
## 1     John  15000          1
## 2    Peter  23400          2
## 3     Abby  26800          3
## 4     Hope  22000          2
## 5   Robert  35000          4
## 6    Emily  45000          7
\end{verbatim}

\paragraph{2. Saving A data frame}\label{saving-a-data-frame}

\subparagraph{File Types:}\label{file-types}

read\_csv() \#read csv into a tibble \#read.csv() reads into a
dataframe\\
read\_csv2() \# semicolon sv into a tibble \#read.csv2() reads into a
dataframe\\
read\_tsv() \#Reads tab-deliminated into a tibble\\
read\_delim() \# reads files with any delimiter\\
read\_fwf() \#reads fixed width files\\
read\_table() \#reads common variation of fixed-width files separated by
white space\\
\textbf{HINT:For more information explore the readr package }

\section{Saving (writing) data is simple in
R}\label{saving-writing-data-is-simple-in-r}

\begin{Shaded}
\begin{Highlighting}[]
\NormalTok{?}\KeywordTok{write.csv}\NormalTok{()}
\KeywordTok{write.csv}\NormalTok{(}\DataTypeTok{x =} \OperatorTok{<}\NormalTok{Name of R object}\OperatorTok{>}\NormalTok{, }\DataTypeTok{file  =} \OperatorTok{<}\StringTok{"C:/temp/my_data.csv"}\OperatorTok{>}\NormalTok{)}
\end{Highlighting}
\end{Shaded}

Lets save our company data to our computer

\begin{Shaded}
\begin{Highlighting}[]
\KeywordTok{write.csv}\NormalTok{(my_seq, }\StringTok{"C:/temp/data.csv"}\NormalTok{)}
\end{Highlighting}
\end{Shaded}

\subsubsection{Section III: Exploring the
Tidyverse!}\label{section-iii-exploring-the-tidyverse}

\paragraph{Install and Load the tidyverse
package}\label{install-and-load-the-tidyverse-package}

\begin{Shaded}
\begin{Highlighting}[]
\KeywordTok{require}\NormalTok{(}\StringTok{"tidyverse"}\NormalTok{)}
\end{Highlighting}
\end{Shaded}

\begin{verbatim}
## Loading required package: tidyverse
\end{verbatim}

\begin{verbatim}
## Warning: package 'tidyverse' was built under R version 3.5.3
\end{verbatim}

\begin{verbatim}
## -- Attaching packages --------------------------------------------------------------------------------------------------------------- tidyverse 1.2.1 --
\end{verbatim}

\begin{verbatim}
## v ggplot2 3.1.1     v purrr   0.3.2
## v tibble  2.1.1     v dplyr   0.8.1
## v tidyr   0.8.3     v stringr 1.4.0
## v readr   1.3.1     v forcats 0.4.0
\end{verbatim}

\begin{verbatim}
## Warning: package 'ggplot2' was built under R version 3.5.3
\end{verbatim}

\begin{verbatim}
## Warning: package 'tibble' was built under R version 3.5.3
\end{verbatim}

\begin{verbatim}
## Warning: package 'tidyr' was built under R version 3.5.3
\end{verbatim}

\begin{verbatim}
## Warning: package 'readr' was built under R version 3.5.2
\end{verbatim}

\begin{verbatim}
## Warning: package 'purrr' was built under R version 3.5.3
\end{verbatim}

\begin{verbatim}
## Warning: package 'dplyr' was built under R version 3.5.3
\end{verbatim}

\begin{verbatim}
## Warning: package 'stringr' was built under R version 3.5.2
\end{verbatim}

\begin{verbatim}
## Warning: package 'forcats' was built under R version 3.5.3
\end{verbatim}

\begin{verbatim}
## -- Conflicts ------------------------------------------------------------------------------------------------------------------ tidyverse_conflicts() --
## x dplyr::filter() masks stats::filter()
## x dplyr::lag()    masks stats::lag()
\end{verbatim}

\begin{Shaded}
\begin{Highlighting}[]
\KeywordTok{require}\NormalTok{(}\StringTok{"gapminder"}\NormalTok{)}
\end{Highlighting}
\end{Shaded}

\begin{verbatim}
## Loading required package: gapminder
\end{verbatim}

\begin{verbatim}
## Warning: package 'gapminder' was built under R version 3.5.3
\end{verbatim}

\paragraph{Explore the Tidyverse}\label{explore-the-tidyverse}

\url{https://www.tidyverse.org/}

R packages only have to be installed once but loaded everytime.\\
Using require is a nice way to make sure every script has the packages
needed which combines install.packages() \& library()

\paragraph{1. Basic Data Exploration}\label{basic-data-exploration}

In this section we will use the gapminder data set
\url{https://www.gapminder.org/}

\subparagraph{Lets assign this to
gapminder}\label{lets-assign-this-to-gapminder}

\begin{Shaded}
\begin{Highlighting}[]
\NormalTok{gapminder <-}\StringTok{ }\NormalTok{gapminder}
\end{Highlighting}
\end{Shaded}

\subparagraph{View our table}\label{view-our-table}

\begin{Shaded}
\begin{Highlighting}[]
\KeywordTok{View}\NormalTok{(gapminder)}
\end{Highlighting}
\end{Shaded}

\subparagraph{Lists the variables}\label{lists-the-variables}

\begin{Shaded}
\begin{Highlighting}[]
\KeywordTok{names}\NormalTok{(gapminder)}
\end{Highlighting}
\end{Shaded}

\begin{verbatim}
## [1] "country"   "continent" "year"      "lifeExp"   "pop"       "gdpPercap"
\end{verbatim}

\subparagraph{Lets Examine the structure of the
data}\label{lets-examine-the-structure-of-the-data}

\begin{Shaded}
\begin{Highlighting}[]
\KeywordTok{str}\NormalTok{(gapminder)}
\end{Highlighting}
\end{Shaded}

\begin{verbatim}
## Classes 'tbl_df', 'tbl' and 'data.frame':    1704 obs. of  6 variables:
##  $ country  : Factor w/ 142 levels "Afghanistan",..: 1 1 1 1 1 1 1 1 1 1 ...
##  $ continent: Factor w/ 5 levels "Africa","Americas",..: 3 3 3 3 3 3 3 3 3 3 ...
##  $ year     : int  1952 1957 1962 1967 1972 1977 1982 1987 1992 1997 ...
##  $ lifeExp  : num  28.8 30.3 32 34 36.1 ...
##  $ pop      : int  8425333 9240934 10267083 11537966 13079460 14880372 12881816 13867957 16317921 22227415 ...
##  $ gdpPercap: num  779 821 853 836 740 ...
\end{verbatim}

This will become very useful when we visualize or analyze data, because
we must make sure our variables are in the appropriate format!!

\subparagraph{Statistical summary of the
data}\label{statistical-summary-of-the-data}

\begin{Shaded}
\begin{Highlighting}[]
\KeywordTok{summary}\NormalTok{(gapminder)}
\end{Highlighting}
\end{Shaded}

\begin{verbatim}
##         country        continent        year         lifeExp     
##  Afghanistan:  12   Africa  :624   Min.   :1952   Min.   :23.60  
##  Albania    :  12   Americas:300   1st Qu.:1966   1st Qu.:48.20  
##  Algeria    :  12   Asia    :396   Median :1980   Median :60.71  
##  Angola     :  12   Europe  :360   Mean   :1980   Mean   :59.47  
##  Argentina  :  12   Oceania : 24   3rd Qu.:1993   3rd Qu.:70.85  
##  Australia  :  12                  Max.   :2007   Max.   :82.60  
##  (Other)    :1632                                                
##       pop              gdpPercap       
##  Min.   :6.001e+04   Min.   :   241.2  
##  1st Qu.:2.794e+06   1st Qu.:  1202.1  
##  Median :7.024e+06   Median :  3531.8  
##  Mean   :2.960e+07   Mean   :  7215.3  
##  3rd Qu.:1.959e+07   3rd Qu.:  9325.5  
##  Max.   :1.319e+09   Max.   :113523.1  
## 
\end{verbatim}

\paragraph{2. Exploring our data
further}\label{exploring-our-data-further}

\textbf{HINT: Understanding how data is indexed is crutial for R
programming}

\subparagraph{Lets look at column 2}\label{lets-look-at-column-2}

\begin{Shaded}
\begin{Highlighting}[]
\NormalTok{gapminder[,}\DecValTok{2}\NormalTok{]}
\end{Highlighting}
\end{Shaded}

\begin{verbatim}
## # A tibble: 1,704 x 1
##    continent
##    <fct>    
##  1 Asia     
##  2 Asia     
##  3 Asia     
##  4 Asia     
##  5 Asia     
##  6 Asia     
##  7 Asia     
##  8 Asia     
##  9 Asia     
## 10 Asia     
## # ... with 1,694 more rows
\end{verbatim}

gapminder{[},2{]}

\subparagraph{Lets look at row 5}\label{lets-look-at-row-5}

\begin{Shaded}
\begin{Highlighting}[]
\NormalTok{gapminder[}\DecValTok{5}\NormalTok{,]}
\end{Highlighting}
\end{Shaded}

\begin{verbatim}
## # A tibble: 1 x 6
##   country     continent  year lifeExp      pop gdpPercap
##   <fct>       <fct>     <int>   <dbl>    <int>     <dbl>
## 1 Afghanistan Asia       1972    36.1 13079460      740.
\end{verbatim}

\subparagraph{What value is in row 5 column
3?}\label{what-value-is-in-row-5-column-3}

\begin{Shaded}
\begin{Highlighting}[]
\NormalTok{gapminder[}\DecValTok{5}\NormalTok{,}\DecValTok{3}\NormalTok{]}
\end{Highlighting}
\end{Shaded}

\begin{verbatim}
## # A tibble: 1 x 1
##    year
##   <int>
## 1  1972
\end{verbatim}

\subparagraph{Based on this idea, we can make more complicated searches.
Lets take the first ten observations and look at the variables:Country
(1), Continent(2), Year (3), and population
(5)}\label{based-on-this-idea-we-can-make-more-complicated-searches.-lets-take-the-first-ten-observations-and-look-at-the-variablescountry-1-continent2-year-3-and-population-5}

\begin{Shaded}
\begin{Highlighting}[]
\NormalTok{gapminder[}\DecValTok{1}\OperatorTok{:}\DecValTok{10}\NormalTok{,}\KeywordTok{c}\NormalTok{(}\DecValTok{1}\OperatorTok{:}\DecValTok{3}\NormalTok{, }\DecValTok{5}\NormalTok{)]}
\end{Highlighting}
\end{Shaded}

\begin{verbatim}
## # A tibble: 10 x 4
##    country     continent  year      pop
##    <fct>       <fct>     <int>    <int>
##  1 Afghanistan Asia       1952  8425333
##  2 Afghanistan Asia       1957  9240934
##  3 Afghanistan Asia       1962 10267083
##  4 Afghanistan Asia       1967 11537966
##  5 Afghanistan Asia       1972 13079460
##  6 Afghanistan Asia       1977 14880372
##  7 Afghanistan Asia       1982 12881816
##  8 Afghanistan Asia       1987 13867957
##  9 Afghanistan Asia       1992 16317921
## 10 Afghanistan Asia       1997 22227415
\end{verbatim}

\subparagraph{What if we want to know the highest
gpdPercap}\label{what-if-we-want-to-know-the-highest-gpdpercap}

\begin{Shaded}
\begin{Highlighting}[]
\KeywordTok{max}\NormalTok{(gapminder}\OperatorTok{$}\NormalTok{gdpPercap)}
\end{Highlighting}
\end{Shaded}

\begin{verbatim}
## [1] 113523.1
\end{verbatim}

\subparagraph{Lets find the row number of the country with the highest
gpdpercap}\label{lets-find-the-row-number-of-the-country-with-the-highest-gpdpercap}

\begin{Shaded}
\begin{Highlighting}[]
\KeywordTok{which.max}\NormalTok{(gapminder}\OperatorTok{$}\NormalTok{gdpPercap)}
\end{Highlighting}
\end{Shaded}

\begin{verbatim}
## [1] 854
\end{verbatim}

\section{Then show me all columns for row that
row}\label{then-show-me-all-columns-for-row-that-row}

\begin{Shaded}
\begin{Highlighting}[]
\NormalTok{gapminder[}\DecValTok{854}\NormalTok{,]}
\end{Highlighting}
\end{Shaded}

\begin{verbatim}
## # A tibble: 1 x 6
##   country continent  year lifeExp    pop gdpPercap
##   <fct>   <fct>     <int>   <dbl>  <int>     <dbl>
## 1 Kuwait  Asia       1957    58.0 212846   113523.
\end{verbatim}

\paragraph{2. The filter verb}\label{the-filter-verb}

The filter verb is used to look at a subset of a data set.\\
Typically you combine filter with a pipe \%\textgreater{}\%

\begin{Shaded}
\begin{Highlighting}[]
\NormalTok{gapminder }\OperatorTok\StringTok{ }
\StringTok{  }\KeywordTok{filter}\NormalTok{(country }\OperatorTok{==}\StringTok{ "United States"}\NormalTok{)}
\end{Highlighting}
\end{Shaded}

\begin{verbatim}
## # A tibble: 12 x 6
##    country       continent  year lifeExp       pop gdpPercap
##    <fct>         <fct>     <int>   <dbl>     <int>     <dbl>
##  1 United States Americas   1952    68.4 157553000    13990.
##  2 United States Americas   1957    69.5 171984000    14847.
##  3 United States Americas   1962    70.2 186538000    16173.
##  4 United States Americas   1967    70.8 198712000    19530.
##  5 United States Americas   1972    71.3 209896000    21806.
##  6 United States Americas   1977    73.4 220239000    24073.
##  7 United States Americas   1982    74.6 232187835    25010.
##  8 United States Americas   1987    75.0 242803533    29884.
##  9 United States Americas   1992    76.1 256894189    32004.
## 10 United States Americas   1997    76.8 272911760    35767.
## 11 United States Americas   2002    77.3 287675526    39097.
## 12 United States Americas   2007    78.2 301139947    42952.
\end{verbatim}

\subparagraph{Multiple conditions}\label{multiple-conditions}

\begin{Shaded}
\begin{Highlighting}[]
\NormalTok{gapminder }\OperatorTok\StringTok{ }
\StringTok{  }\KeywordTok{filter}\NormalTok{(year }\OperatorTok{==}\StringTok{ }\DecValTok{2007}\NormalTok{, country }\OperatorTok{==}\StringTok{ "United States"}\NormalTok{)}
\end{Highlighting}
\end{Shaded}

\begin{verbatim}
## # A tibble: 1 x 6
##   country       continent  year lifeExp       pop gdpPercap
##   <fct>         <fct>     <int>   <dbl>     <int>     <dbl>
## 1 United States Americas   2007    78.2 301139947    42952.
\end{verbatim}

\paragraph{The arrange verb
----------------------------------}\label{the-arrange-verb--}

Used for sorting data by ascending or descending condition\\
\#\#\#\#\# Ascending Order

\begin{Shaded}
\begin{Highlighting}[]
\NormalTok{gapminder }\OperatorTok\StringTok{ }
\StringTok{  }\KeywordTok{arrange}\NormalTok{(gdpPercap)}
\end{Highlighting}
\end{Shaded}

\begin{verbatim}
## # A tibble: 1,704 x 6
##    country          continent  year lifeExp      pop gdpPercap
##    <fct>            <fct>     <int>   <dbl>    <int>     <dbl>
##  1 Congo, Dem. Rep. Africa     2002    45.0 55379852      241.
##  2 Congo, Dem. Rep. Africa     2007    46.5 64606759      278.
##  3 Lesotho          Africa     1952    42.1   748747      299.
##  4 Guinea-Bissau    Africa     1952    32.5   580653      300.
##  5 Congo, Dem. Rep. Africa     1997    42.6 47798986      312.
##  6 Eritrea          Africa     1952    35.9  1438760      329.
##  7 Myanmar          Asia       1952    36.3 20092996      331 
##  8 Lesotho          Africa     1957    45.0   813338      336.
##  9 Burundi          Africa     1952    39.0  2445618      339.
## 10 Eritrea          Africa     1957    38.0  1542611      344.
## # ... with 1,694 more rows
\end{verbatim}

\subparagraph{Descending order}\label{descending-order}

\begin{Shaded}
\begin{Highlighting}[]
\NormalTok{gapminder }\OperatorTok\StringTok{ }
\StringTok{  }\KeywordTok{arrange}\NormalTok{(}\KeywordTok{desc}\NormalTok{(gdpPercap))}
\end{Highlighting}
\end{Shaded}

\begin{verbatim}
## # A tibble: 1,704 x 6
##    country   continent  year lifeExp     pop gdpPercap
##    <fct>     <fct>     <int>   <dbl>   <int>     <dbl>
##  1 Kuwait    Asia       1957    58.0  212846   113523.
##  2 Kuwait    Asia       1972    67.7  841934   109348.
##  3 Kuwait    Asia       1952    55.6  160000   108382.
##  4 Kuwait    Asia       1962    60.5  358266    95458.
##  5 Kuwait    Asia       1967    64.6  575003    80895.
##  6 Kuwait    Asia       1977    69.3 1140357    59265.
##  7 Norway    Europe     2007    80.2 4627926    49357.
##  8 Kuwait    Asia       2007    77.6 2505559    47307.
##  9 Singapore Asia       2007    80.0 4553009    47143.
## 10 Norway    Europe     2002    79.0 4535591    44684.
## # ... with 1,694 more rows
\end{verbatim}

\subparagraph{Combining verbs}\label{combining-verbs}

\begin{Shaded}
\begin{Highlighting}[]
\NormalTok{gapminder }\OperatorTok\StringTok{ }
\StringTok{  }\KeywordTok{filter}\NormalTok{(year }\OperatorTok{==}\StringTok{ }\DecValTok{2007}\NormalTok{) }\OperatorTok\StringTok{ }
\StringTok{  }\KeywordTok{arrange}\NormalTok{(gdpPercap)}
\end{Highlighting}
\end{Shaded}

\begin{verbatim}
## # A tibble: 142 x 6
##    country                  continent  year lifeExp      pop gdpPercap
##    <fct>                    <fct>     <int>   <dbl>    <int>     <dbl>
##  1 Congo, Dem. Rep.         Africa     2007    46.5 64606759      278.
##  2 Liberia                  Africa     2007    45.7  3193942      415.
##  3 Burundi                  Africa     2007    49.6  8390505      430.
##  4 Zimbabwe                 Africa     2007    43.5 12311143      470.
##  5 Guinea-Bissau            Africa     2007    46.4  1472041      579.
##  6 Niger                    Africa     2007    56.9 12894865      620.
##  7 Eritrea                  Africa     2007    58.0  4906585      641.
##  8 Ethiopia                 Africa     2007    52.9 76511887      691.
##  9 Central African Republic Africa     2007    44.7  4369038      706.
## 10 Gambia                   Africa     2007    59.4  1688359      753.
## # ... with 132 more rows
\end{verbatim}

\paragraph{The mutate verb}\label{the-mutate-verb}

Change or Add variables to a data set

\subparagraph{Change a variable}\label{change-a-variable}

\begin{Shaded}
\begin{Highlighting}[]
\NormalTok{gapminder }\OperatorTok\StringTok{ }
\StringTok{  }\KeywordTok{mutate}\NormalTok{(}\DataTypeTok{pop =}\NormalTok{ pop}\OperatorTok{/}\DecValTok{1000000}\NormalTok{)}
\end{Highlighting}
\end{Shaded}

\begin{verbatim}
## # A tibble: 1,704 x 6
##    country     continent  year lifeExp   pop gdpPercap
##    <fct>       <fct>     <int>   <dbl> <dbl>     <dbl>
##  1 Afghanistan Asia       1952    28.8  8.43      779.
##  2 Afghanistan Asia       1957    30.3  9.24      821.
##  3 Afghanistan Asia       1962    32.0 10.3       853.
##  4 Afghanistan Asia       1967    34.0 11.5       836.
##  5 Afghanistan Asia       1972    36.1 13.1       740.
##  6 Afghanistan Asia       1977    38.4 14.9       786.
##  7 Afghanistan Asia       1982    39.9 12.9       978.
##  8 Afghanistan Asia       1987    40.8 13.9       852.
##  9 Afghanistan Asia       1992    41.7 16.3       649.
## 10 Afghanistan Asia       1997    41.8 22.2       635.
## # ... with 1,694 more rows
\end{verbatim}

\subparagraph{Add a new variable called
gdp}\label{add-a-new-variable-called-gdp}

\begin{Shaded}
\begin{Highlighting}[]
\NormalTok{gapminder }\OperatorTok\StringTok{ }
\StringTok{  }\KeywordTok{mutate}\NormalTok{(}\DataTypeTok{gdp =}\NormalTok{ gdpPercap }\OperatorTok{*}\StringTok{ }\NormalTok{pop)}
\end{Highlighting}
\end{Shaded}

\begin{verbatim}
## # A tibble: 1,704 x 7
##    country     continent  year lifeExp      pop gdpPercap          gdp
##    <fct>       <fct>     <int>   <dbl>    <int>     <dbl>        <dbl>
##  1 Afghanistan Asia       1952    28.8  8425333      779.  6567086330.
##  2 Afghanistan Asia       1957    30.3  9240934      821.  7585448670.
##  3 Afghanistan Asia       1962    32.0 10267083      853.  8758855797.
##  4 Afghanistan Asia       1967    34.0 11537966      836.  9648014150.
##  5 Afghanistan Asia       1972    36.1 13079460      740.  9678553274.
##  6 Afghanistan Asia       1977    38.4 14880372      786. 11697659231.
##  7 Afghanistan Asia       1982    39.9 12881816      978. 12598563401.
##  8 Afghanistan Asia       1987    40.8 13867957      852. 11820990309.
##  9 Afghanistan Asia       1992    41.7 16317921      649. 10595901589.
## 10 Afghanistan Asia       1997    41.8 22227415      635. 14121995875.
## # ... with 1,694 more rows
\end{verbatim}

\subsubsection{Combine all three verbs}\label{combine-all-three-verbs}

\begin{Shaded}
\begin{Highlighting}[]
\NormalTok{gapminder }\OperatorTok\StringTok{ }
\StringTok{  }\KeywordTok{mutate}\NormalTok{(}\DataTypeTok{gdp =}\NormalTok{ gdpPercap }\OperatorTok{*}\StringTok{ }\NormalTok{pop) }\OperatorTok\StringTok{ }
\StringTok{  }\KeywordTok{filter}\NormalTok{(year }\OperatorTok{==}\StringTok{ }\DecValTok{2007}\NormalTok{) }\OperatorTok\StringTok{ }
\StringTok{  }\KeywordTok{arrange}\NormalTok{(}\KeywordTok{desc}\NormalTok{(gdp))}
\end{Highlighting}
\end{Shaded}

\begin{verbatim}
## # A tibble: 142 x 7
##    country        continent  year lifeExp        pop gdpPercap     gdp
##    <fct>          <fct>     <int>   <dbl>      <int>     <dbl>   <dbl>
##  1 United States  Americas   2007    78.2  301139947    42952. 1.29e13
##  2 China          Asia       2007    73.0 1318683096     4959. 6.54e12
##  3 Japan          Asia       2007    82.6  127467972    31656. 4.04e12
##  4 India          Asia       2007    64.7 1110396331     2452. 2.72e12
##  5 Germany        Europe     2007    79.4   82400996    32170. 2.65e12
##  6 United Kingdom Europe     2007    79.4   60776238    33203. 2.02e12
##  7 France         Europe     2007    80.7   61083916    30470. 1.86e12
##  8 Brazil         Americas   2007    72.4  190010647     9066. 1.72e12
##  9 Italy          Europe     2007    80.5   58147733    28570. 1.66e12
## 10 Mexico         Americas   2007    76.2  108700891    11978. 1.30e12
## # ... with 132 more rows
\end{verbatim}

\paragraph{The Summarize Verb}\label{the-summarize-verb}

\subparagraph{Summarize entire data
set}\label{summarize-entire-data-set}

Returns the mean of all rows (one value)

\begin{Shaded}
\begin{Highlighting}[]
\NormalTok{gapminder }\OperatorTok\StringTok{ }
\StringTok{  }\KeywordTok{summarize}\NormalTok{(}\DataTypeTok{meanLifeExp =} \KeywordTok{mean}\NormalTok{(lifeExp))}
\end{Highlighting}
\end{Shaded}

\begin{verbatim}
## # A tibble: 1 x 1
##   meanLifeExp
##         <dbl>
## 1        59.5
\end{verbatim}

\subparagraph{What if we want to return the mean life exp just for
2007}\label{what-if-we-want-to-return-the-mean-life-exp-just-for-2007}

\begin{Shaded}
\begin{Highlighting}[]
\NormalTok{gapminder }\OperatorTok\StringTok{ }
\StringTok{  }\KeywordTok{filter}\NormalTok{(year }\OperatorTok{==}\StringTok{ }\DecValTok{2007}\NormalTok{) }\OperatorTok\StringTok{ }
\StringTok{  }\KeywordTok{summarize}\NormalTok{(}\DataTypeTok{meanLifeExp =} \KeywordTok{mean}\NormalTok{(lifeExp))}
\end{Highlighting}
\end{Shaded}

\begin{verbatim}
## # A tibble: 1 x 1
##   meanLifeExp
##         <dbl>
## 1        67.0
\end{verbatim}

\subparagraph{Creating multiple
Summaries}\label{creating-multiple-summaries}

\begin{Shaded}
\begin{Highlighting}[]
\NormalTok{gapminder }\OperatorTok\StringTok{ }
\StringTok{  }\KeywordTok{filter}\NormalTok{(year }\OperatorTok{==}\StringTok{ }\DecValTok{2007}\NormalTok{) }\OperatorTok\StringTok{ }
\StringTok{  }\KeywordTok{summarize}\NormalTok{(}\DataTypeTok{meanLifeExp =} \KeywordTok{mean}\NormalTok{(lifeExp),}
            \DataTypeTok{totalPop =} \KeywordTok{sum}\NormalTok{(pop))}
\end{Highlighting}
\end{Shaded}

\begin{verbatim}
## Warning in summarise_impl(.data, dots, environment(), caller_env()):
## integer overflow - use sum(as.numeric(.))
\end{verbatim}

\begin{verbatim}
## # A tibble: 1 x 2
##   meanLifeExp totalPop
##         <dbl>    <int>
## 1        67.0       NA
\end{verbatim}

\textbf{HINT: What data type is pop? Use str(gapminder)}

\subparagraph{Convert pop to a numeric data type instead of an
integer}\label{convert-pop-to-a-numeric-data-type-instead-of-an-integer}

\begin{Shaded}
\begin{Highlighting}[]
\NormalTok{gapminder}\OperatorTok{$}\NormalTok{pop <-}\StringTok{ }\KeywordTok{as.numeric}\NormalTok{(gapminder}\OperatorTok{$}\NormalTok{pop)}
\end{Highlighting}
\end{Shaded}

\paragraph{The group\_by Verb}\label{the-group_by-verb}

The group\_by verb is useful for creating aggregated groups, especially
when combined with the summarize function

\subparagraph{Summarize by each unique
year}\label{summarize-by-each-unique-year}

\begin{Shaded}
\begin{Highlighting}[]
\NormalTok{gapminder }\OperatorTok\StringTok{ }
\StringTok{  }\KeywordTok{group_by}\NormalTok{(year) }\OperatorTok\StringTok{ }
\StringTok{  }\KeywordTok{summarize}\NormalTok{(}\DataTypeTok{meanLifeExp =} \KeywordTok{mean}\NormalTok{(lifeExp),}
            \DataTypeTok{totalPop =} \KeywordTok{sum}\NormalTok{(pop))}
\end{Highlighting}
\end{Shaded}

\begin{verbatim}
## # A tibble: 12 x 3
##     year meanLifeExp   totalPop
##    <int>       <dbl>      <dbl>
##  1  1952        49.1 2406957150
##  2  1957        51.5 2664404580
##  3  1962        53.6 2899782974
##  4  1967        55.7 3217478384
##  5  1972        57.6 3576977158
##  6  1977        59.6 3930045807
##  7  1982        61.5 4289436840
##  8  1987        63.2 4691477418
##  9  1992        64.2 5110710260
## 10  1997        65.0 5515204472
## 11  2002        65.7 5886977579
## 12  2007        67.0 6251013179
\end{verbatim}

\subparagraph{Summarize data from 2007 by
continent}\label{summarize-data-from-2007-by-continent}

\begin{Shaded}
\begin{Highlighting}[]
\NormalTok{gapminder }\OperatorTok\StringTok{ }
\StringTok{  }\KeywordTok{filter}\NormalTok{(year }\OperatorTok{==}\StringTok{ }\DecValTok{2007}\NormalTok{) }\OperatorTok\StringTok{ }
\StringTok{  }\KeywordTok{group_by}\NormalTok{(continent) }\OperatorTok\StringTok{ }
\StringTok{  }\KeywordTok{summarize}\NormalTok{(}\DataTypeTok{meanLifeExp =} \KeywordTok{mean}\NormalTok{(lifeExp),}
            \DataTypeTok{totalPop =} \KeywordTok{sum}\NormalTok{(pop))}
\end{Highlighting}
\end{Shaded}

\begin{verbatim}
## # A tibble: 5 x 3
##   continent meanLifeExp   totalPop
##   <fct>           <dbl>      <dbl>
## 1 Africa           54.8  929539692
## 2 Americas         73.6  898871184
## 3 Asia             70.7 3811953827
## 4 Europe           77.6  586098529
## 5 Oceania          80.7   24549947
\end{verbatim}

\subparagraph{What if we want to summarize by continent over all
years?}\label{what-if-we-want-to-summarize-by-continent-over-all-years}

\textbf{HINT: Simply add an additional arguement to the group\_by verb}

\begin{Shaded}
\begin{Highlighting}[]
\NormalTok{gapminder }\OperatorTok\StringTok{ }
\StringTok{  }\KeywordTok{group_by}\NormalTok{(year, continent) }\OperatorTok\StringTok{ }
\StringTok{  }\KeywordTok{summarize}\NormalTok{(}\DataTypeTok{meanLifeExp =} \KeywordTok{mean}\NormalTok{(lifeExp),}
            \DataTypeTok{totalPop =} \KeywordTok{sum}\NormalTok{(pop))}
\end{Highlighting}
\end{Shaded}

\begin{verbatim}
## # A tibble: 60 x 4
## # Groups:   year [12]
##     year continent meanLifeExp   totalPop
##    <int> <fct>           <dbl>      <dbl>
##  1  1952 Africa           39.1  237640501
##  2  1952 Americas         53.3  345152446
##  3  1952 Asia             46.3 1395357351
##  4  1952 Europe           64.4  418120846
##  5  1952 Oceania          69.3   10686006
##  6  1957 Africa           41.3  264837738
##  7  1957 Americas         56.0  386953916
##  8  1957 Asia             49.3 1562780599
##  9  1957 Europe           66.7  437890351
## 10  1957 Oceania          70.3   11941976
## # ... with 50 more rows
\end{verbatim}

\subsubsection{Section III Tasks}\label{section-iii-tasks}

Answer the following questions using the mtcars dataset

\begin{Shaded}
\begin{Highlighting}[]
\NormalTok{mtcars <-}\StringTok{ }\NormalTok{mtcars}
\end{Highlighting}
\end{Shaded}

\subparagraph{Task 3A. find the median mpg \& wt for each group of
cylinders}\label{task-3a.-find-the-median-mpg-wt-for-each-group-of-cylinders}

\begin{Shaded}
\begin{Highlighting}[]
\NormalTok{mtcars }\OperatorTok\StringTok{ }
\StringTok{  }\KeywordTok{group_by}\NormalTok{(cyl) }\OperatorTok\StringTok{ }
\StringTok{  }\KeywordTok{summarize}\NormalTok{(}\DataTypeTok{median_mpg =} \KeywordTok{median}\NormalTok{(mpg),}
            \DataTypeTok{median_wt =} \KeywordTok{median}\NormalTok{(wt))}
\end{Highlighting}
\end{Shaded}

\begin{verbatim}
## # A tibble: 3 x 3
##     cyl median_mpg median_wt
##   <dbl>      <dbl>     <dbl>
## 1     4       26        2.2 
## 2     6       19.7      3.22
## 3     8       15.2      3.76
\end{verbatim}

mtcars \%\textgreater{}\% group\_by(cyl) \%\textgreater{}\%
summarize(median\_mpg = median(mpg), median\_wt = median(wt))

\subparagraph{Task 3B. find the mean of hp and drat for each group of
gear and cyl
\&}\label{task-3b.-find-the-mean-of-hp-and-drat-for-each-group-of-gear-and-cyl}

\begin{verbatim}
      #find the ratio between mean hp and mean drat
      
\end{verbatim}

\begin{Shaded}
\begin{Highlighting}[]
\NormalTok{mtcars }\OperatorTok\StringTok{ }
\StringTok{  }\KeywordTok{group_by}\NormalTok{(cyl,gear) }\OperatorTok\StringTok{ }
\StringTok{  }\KeywordTok{summarize}\NormalTok{(}\DataTypeTok{mean_drat =} \KeywordTok{mean}\NormalTok{(drat),}
            \DataTypeTok{mean_hp =} \KeywordTok{mean}\NormalTok{(hp),}
            \DataTypeTok{hp_drat_ratio =}\NormalTok{ mean_hp}\OperatorTok{/}\NormalTok{mean_drat)}
\end{Highlighting}
\end{Shaded}

\begin{verbatim}
## # A tibble: 8 x 5
## # Groups:   cyl [3]
##     cyl  gear mean_drat mean_hp hp_drat_ratio
##   <dbl> <dbl>     <dbl>   <dbl>         <dbl>
## 1     4     3      3.7      97           26.2
## 2     4     4      4.11     76           18.5
## 3     4     5      4.1     102           24.9
## 4     6     3      2.92    108.          36.8
## 5     6     4      3.91    116.          29.8
## 6     6     5      3.62    175           48.3
## 7     8     3      3.12    194.          62.2
## 8     8     5      3.88    300.          77.2
\end{verbatim}

\subsubsection{Section IV: Programming with
R}\label{section-iv-programming-with-r}

\paragraph{Writing Functions}\label{writing-functions}

Functions are like black boxes but in this case you are creating the box

\subparagraph{Function recipe}\label{function-recipe}

my\_fun \textless{}- function(arg1, arg2) \{ body \}

\subparagraph{Example \#1}\label{example-1}

Lets write a function that outputs 3x the input, x, and call the
function triple()

\begin{Shaded}
\begin{Highlighting}[]
\NormalTok{triple <-}\StringTok{ }\ControlFlowTok{function}\NormalTok{(x) \{}
\NormalTok{  x}\OperatorTok{*}\DecValTok{3}
\NormalTok{\}}
\end{Highlighting}
\end{Shaded}

\subparagraph{Test the triple function}\label{test-the-triple-function}

\textbf{Hint: the last argument of the r function is automatically
returned}

\begin{Shaded}
\begin{Highlighting}[]
\KeywordTok{triple}\NormalTok{(}\DecValTok{6}\NormalTok{)}
\end{Highlighting}
\end{Shaded}

\begin{verbatim}
## [1] 18
\end{verbatim}

\begin{Shaded}
\begin{Highlighting}[]
\KeywordTok{triple}\NormalTok{(}\DecValTok{10}\NormalTok{)}
\end{Highlighting}
\end{Shaded}

\begin{verbatim}
## [1] 30
\end{verbatim}

\begin{Shaded}
\begin{Highlighting}[]
\KeywordTok{triple}\NormalTok{(}\OperatorTok{-}\FloatTok{1.3}\NormalTok{)}
\end{Highlighting}
\end{Shaded}

\begin{verbatim}
## [1] -3.9
\end{verbatim}

\subparagraph{Example \#2}\label{example-2}

Lets write a function that outputs the area of circle. In this case we
only need one input, the radius, since pi constant.

\begin{Shaded}
\begin{Highlighting}[]
\NormalTok{AoC <-}\StringTok{ }\ControlFlowTok{function}\NormalTok{(x) \{}
\NormalTok{  pi}\OperatorTok{*}\NormalTok{(x)}\OperatorTok{^}\DecValTok{2}
\NormalTok{\}}
\end{Highlighting}
\end{Shaded}

\subparagraph{Test AoC function}\label{test-aoc-function}

\begin{Shaded}
\begin{Highlighting}[]
\KeywordTok{AoC}\NormalTok{(}\DecValTok{4}\NormalTok{)}
\end{Highlighting}
\end{Shaded}

\begin{verbatim}
## [1] 50.24
\end{verbatim}

\begin{Shaded}
\begin{Highlighting}[]
\KeywordTok{AoC}\NormalTok{(}\DecValTok{10}\NormalTok{)}
\end{Highlighting}
\end{Shaded}

\begin{verbatim}
## [1] 314
\end{verbatim}

\begin{Shaded}
\begin{Highlighting}[]
\KeywordTok{AoC}\NormalTok{(}\FloatTok{2.35}\NormalTok{)}
\end{Highlighting}
\end{Shaded}

\begin{verbatim}
## [1] 17.34065
\end{verbatim}

\paragraph{Apply Family}\label{apply-family}

Running loops is extremely common in programming. One power aspect to R
is that it has built in functions that make running loops easier than
using traditional for and while loops. These functions are contained in
the apply family, which we will briefly cover.

\subparagraph{lapply \& sapply: Use your area of a circle function to
calculate a the area for a series of
radii}\label{lapply-sapply-use-your-area-of-a-circle-function-to-calculate-a-the-area-for-a-series-of-radii}

\begin{Shaded}
\begin{Highlighting}[]
\NormalTok{?}\KeywordTok{lapply}\NormalTok{()}
\end{Highlighting}
\end{Shaded}

\begin{verbatim}
## starting httpd help server ... done
\end{verbatim}

\begin{Shaded}
\begin{Highlighting}[]
\NormalTok{?}\KeywordTok{sapply}\NormalTok{()}
\end{Highlighting}
\end{Shaded}

\subparagraph{Create a list of radii}\label{create-a-list-of-radii}

\begin{Shaded}
\begin{Highlighting}[]
\NormalTok{radii <-}\StringTok{ }\KeywordTok{list}\NormalTok{(}\FloatTok{2.37}\NormalTok{,}\FloatTok{2.49}\NormalTok{,}\FloatTok{2.18}\NormalTok{,}\FloatTok{2.22}\NormalTok{,}\FloatTok{2.47}\NormalTok{,}\FloatTok{2.32}\NormalTok{)}
\end{Highlighting}
\end{Shaded}

\subparagraph{Make sure your AoC funcion is
loaded}\label{make-sure-your-aoc-funcion-is-loaded}

\begin{Shaded}
\begin{Highlighting}[]
\NormalTok{AoC <-}\StringTok{ }\ControlFlowTok{function}\NormalTok{(x) \{}
\NormalTok{  pi}\OperatorTok{*}\NormalTok{(x)}\OperatorTok{^}\DecValTok{2}
\NormalTok{\}}
\end{Highlighting}
\end{Shaded}

\subparagraph{test lapply on your list}\label{test-lapply-on-your-list}

\begin{Shaded}
\begin{Highlighting}[]
\KeywordTok{lapply}\NormalTok{(radii, AoC)}
\end{Highlighting}
\end{Shaded}

\begin{verbatim}
## [[1]]
## [1] 17.63707
## 
## [[2]]
## [1] 19.46831
## 
## [[3]]
## [1] 14.92254
## 
## [[4]]
## [1] 15.47518
## 
## [[5]]
## [1] 19.15683
## 
## [[6]]
## [1] 16.90074
\end{verbatim}

**HINT: using the unlist() function is commone because the lapply's
output isn't as intutive.

\begin{Shaded}
\begin{Highlighting}[]
\KeywordTok{unlist}\NormalTok{(}\KeywordTok{lapply}\NormalTok{(radii, AoC))}
\end{Highlighting}
\end{Shaded}

\begin{verbatim}
## [1] 17.63707 19.46831 14.92254 15.47518 19.15683 16.90074
\end{verbatim}

\subparagraph{sapply simplifies this task because in our case all
objects are the same data
type.}\label{sapply-simplifies-this-task-because-in-our-case-all-objects-are-the-same-data-type.}

\begin{Shaded}
\begin{Highlighting}[]
\KeywordTok{sapply}\NormalTok{(radii, AoC)}
\end{Highlighting}
\end{Shaded}

\begin{verbatim}
## [1] 17.63707 19.46831 14.92254 15.47518 19.15683 16.90074
\end{verbatim}

\subsubsection{OPTIONAL: Exploring
Loops}\label{optional-exploring-loops}

\paragraph{While Loops}\label{while-loops}

\begin{Shaded}
\begin{Highlighting}[]
\ControlFlowTok{while}\NormalTok{ (condition) \{}
\NormalTok{  expression}
\NormalTok{\}}
\end{Highlighting}
\end{Shaded}

\subparagraph{Counter Example}\label{counter-example}

Lets set a counter (ctr) to run while ctr is less than 7 and print ctr
is set to ``counter''

\begin{Shaded}
\begin{Highlighting}[]
\NormalTok{ctr <-}\StringTok{ }\DecValTok{1}

\ControlFlowTok{while}\NormalTok{( ctr }\OperatorTok{<=}\StringTok{ }\DecValTok{7}\NormalTok{) \{}
  \KeywordTok{print}\NormalTok{(}\KeywordTok{paste}\NormalTok{(}\StringTok{"ctr is set to"}\NormalTok{, ctr))}
\NormalTok{  ctr <-}\StringTok{ }\NormalTok{ctr }\OperatorTok{+}\StringTok{ }\DecValTok{1}
\NormalTok{\}}
\end{Highlighting}
\end{Shaded}

\begin{verbatim}
## [1] "ctr is set to 1"
## [1] "ctr is set to 2"
## [1] "ctr is set to 3"
## [1] "ctr is set to 4"
## [1] "ctr is set to 5"
## [1] "ctr is set to 6"
## [1] "ctr is set to 7"
\end{verbatim}

\paragraph{For Loops}\label{for-loops}

\begin{Shaded}
\begin{Highlighting}[]
\ControlFlowTok{for}\NormalTok{ (var }\ControlFlowTok{in}\NormalTok{ seq)\{}
\NormalTok{  expression}
\NormalTok{\}}
\end{Highlighting}
\end{Shaded}

\subparagraph{Cities Example}\label{cities-example}

\begin{Shaded}
\begin{Highlighting}[]
\NormalTok{cities  <-}\StringTok{ }\KeywordTok{c}\NormalTok{(}\StringTok{"New York"}\NormalTok{, }\StringTok{"Paris"}\NormalTok{, }\StringTok{"London"}\NormalTok{,}
             \StringTok{"Tokyo"}\NormalTok{, }\StringTok{"Rio de Janeiro"}\NormalTok{, }\StringTok{"Cape Town"}\NormalTok{)}

\ControlFlowTok{for}\NormalTok{ (city }\ControlFlowTok{in}\NormalTok{ cities)\{}
  \KeywordTok{print}\NormalTok{(city)}
\NormalTok{\}}
\end{Highlighting}
\end{Shaded}


\end{document}
